\documentclass{article}
\usepackage{listings}
\author{cryptochangements}
\date{\today}

\begin{document}
\noindent\Large WWHM Difficulty Adjustment Algorithm in Masari

\small\noindent Author: Cryptochangements

\noindent\today
\begin{abstract}
This paper details the
cryptocurrency Masari's implementation
and transition to a Weighted-Weighted
Harmonic Mean difficulty adjustment
algorithm to smoothly adjust its difficulty
target so that it maintains a steady 2 minute
block time. The original idea for this
algorithm comes from Tom Harold and
was modified by Scott Roberts. The original C++
implementation was done by Thaer
Khawaja of the Masari Core Team. This
implementation has since been adopted
by several other cryptocurrency projects
and has since paved the way for other
difficulty adjustment algorithms such as
LWMA.
\end{abstract}

\section{Introduction}
Cryptocurrencies need to
constantly adjust their difficulty in order to match the ever changing size of the
network. However, the original
implementations of cryptocurrency
difficulty adjustment algorithms used in
older cryptocurrencies such as Bitcoin Core or Bytecoin,
the original CryptoNote reference
implementation, are rather naive and
adjust slowly. Because of this slow
adjustment, an attacker with a large
portion of the network's hashrate can take
advantage of this latency by mining with a
large amount of compute power such that
the current network difficulty is easy for
them and can flash mine several blocks in
rapid succession until difficulty finally
adjusts. Masari suffered this kind of
attack in its early days of development
and as a result, the Masari developers
worked with Zawy to develop a Weighted-Weighted Harmonic Mean (WWHM)
difficulty adjustment algorithm which
would allow Masari to adjust its difficulty
much quicker and prevent this kind of
attack.

\section{WWHM In Theory}
A Weighted-Weighted
Harmonic Mean based algorithm weighs
newer solved blocks more heavily than
older solved blocks when calculating
difficulty to provide a more accurate
estimation of the network's conditions in
the near future.
\subsection{The Algorithm}
The only data
from the blockchain that the algorithm
requires is the timestamps ($T_{i}$) and
difficulties ($D_{i}$) from the most recent
blocks within a specified block window (w)
on the blockchain. In Masari $w = 60$ but it can be changed based on the necessities of the users of this algorithm. The target in seconds (c), in Masari's case 120, is also a parameter of this function. The first step is to get
the solvetimes ($S_{i} | i = 1 ... w$) for the blocks
within the block window which can be
expressed as ($S_{i}$ = $T_{i}$ - $T_{i-1}$). Next the sum of the weighted solvetimes (H) is calculated. Solvetimes are weighted by multiplying the solvetime by its index in the series so that later elements in the series are weighted heavier. This can be expressed as: 
$$H = \sum_{i=1}^{w} S_{i}\cdot i$$
The constant k is also used which will be defined as $k = (w+1)/2\cdot c$ .
The difficulties within the last w blocks ($d$) also need to be summed such that $$d = \sum_{i=1}^{w}D_i$$

Finally, the next difficulty can be calculated by multiplying the sum of the previous difficulties by k and then dividing by the harmonic mean.
 $$d\cdot k / H$$
 
\end{document}
